\smalltitle{سوال 3}\\
دقت کنید اگر دو بردار
$a$ و $b$
بر هم عمود باشند می‌دانیم که
$||a + b||^2 = ||a||^2 + ||b||^2$.
از آنجا که
$v$ و $u$
بر هم عمودند داریم
$v \cdot u = 0$.
همچنین دقت کنید که اگر
$u + v$ و $u - v$
بر هم عمود باشند داریم:
\begin{gather*}
    |u + v + u - v|^2 = 4|u|^2 = |u + v|^2 + |u - v|^2 = |u|^2 + |v|^2 + |u|^2 + |v|^2 = 2|u|^2 + 2|v|^2\\
    \implies 2|u|^2 - 2|v|^2 = 0 \implies |u|^2 - |v|^2 = 0 \implies |u|^2 = |v|^2 \implies |u| = |v|
\end{gather*}