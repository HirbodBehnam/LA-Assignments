\smalltitle{سوال 1}
\begin{enumerate}[wide, labelwidth=!, labelindent=0pt]
    \item درست است.
دقت کنید که می‌توان فضای
$V$
را به صورت
$\sum a_i v_i$
می‌توان نوشت. حال دقت کنید که برای پایه‌های داده شده داریم:
\begin{gather*}
    a_1 v_1 + a_1 v_2 + a_2 v_2 + a_2 v_3 + a_3 v_3 + a_3 v_4 + a_4 v_4 = 
    a_1 v_1 + (a_1 + a_2) v_2 + (a_2 + a_3) v_3 + (a_3 + a_4) v_4
\end{gather*}
همان طور که مشاهده می‌شود عملا با یک سری تغییر متغیر
(مثلا
$a'_2 = a_1 + a_2$
)
می‌توان به همان ترکیب خطی قبل رسید.
    \item فرض سوال نشان می‌دهد که
\lr{span}
این فضا شامل
$v_3, v_4$
نیست چرا که اگر بودند یکی از عضو‌های فضا خودشان می‌شدند که این برخلاف فرض است.
حال سوال دیگری که ممکن است پیش بیاید این است که آیا ممکن است که برداری مثل
$v_5$
نیز پایه‌ی
$U$
باشد؟ جواب این سوال خیر است. چرا که دقت کنید
$U \subset V$
است. این نشان می‌دهد که اگر
$v_5$
قرار باشد پایه‌ی
$U$
باشد، آنگاه باید پایه‌‌ی
$V$
نیز باشد که نیست.
پس این عبارت درست است و فقط
$v_1, v_2$
پایه‌های
$U$
هستند.
\end{enumerate}