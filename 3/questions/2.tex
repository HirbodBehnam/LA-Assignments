\smalltitle{سوال 2}\\
از استقرا استفاده می‌کنیم. برای حالت
$m=1$
داریم کافی است که بردار
$w = v_1$
را در نظر بگیریم. دقت کنید که
$v_1 \neq 0$
است چرا که مستقل خطی است و نمی‌تواند صفر باشد.
در این صورت حتما
$\langle v_1, v_1 \rangle = ||v_1||^2 > 0$
است.
حال استقرا می‌زنیم. فرض کنید که به کمک گرام اشمیت پایه‌های
$\lbrace u_1, \cdots u_m \rbrace$ را از $\lbrace v_1, \cdots v_m \rbrace$
استخراج می‌کنیم. حال می‌دانیم طبق استقرا که
$\forall i \in [1, m-1] ~ \langle w, v_i \rangle > 0$.
حال می‌خواهیم این عبارت را ثابت کنیم:
$\forall i \in [1, m] ~ \langle w + \alpha u_m, v_i \rangle > 0$.
به صورت کلی طبق تعریف ضرب داخلی داریم
$\langle w + \alpha u_m, v_i \rangle = \langle w , v_i \rangle + \alpha \langle u_m, v_i \rangle$.
حال در صورتی که
$i = m$
باشد داریم:
\begin{align*}
    \langle w , v_m \rangle + \alpha \langle u_m, v_m \rangle &> 0\\
    \alpha &> -\frac{\langle w , v_m \rangle}{\langle u_m, v_m \rangle}
\end{align*}
نکته‌ای که در اینجا وجود دارد این است که
$\langle u_m, v_m \rangle$
برابر صفر نیست چرا که
$v_m$
مستقل خطی از
$\lbrace v_1, \cdots, v_{m-1} \rbrace$
است.
حال در صورتی که
$i \neq m$
باشد داریم:
\begin{gather*}
    \langle w , v_i \rangle + \alpha \langle u_m, v_i \rangle =
    \langle w , v_i \rangle + \alpha \langle u_m, \sum_{j=1}^{i} \beta_j u_j \rangle =
    \langle w , v_i \rangle > 0
\end{gather*}
در ابتدا دقت کنید که
$\langle u_m, \sum_{j=1}^{i} \beta_j u_j \rangle$
به خاطر این صفر شد که
$i < m$
است و اصلا
$u_m$
در این سیگما وجود ندارد!
همچنین در نظر داشته باشید که طبق استقرا توانستیم بنویسیم که
$\langle w , v_i \rangle > 0$
است.
