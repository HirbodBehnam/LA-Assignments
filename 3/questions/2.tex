\smalltitle{سوال 2}\\
از استقرا استفاده می‌کنیم. برای حالت
$m=1$
داریم کافی است که بردار
$w = v_1$
را در نظر بگیریم. دقت کنید که
$v_1 \neq 0$
است چرا که مستقل خطی است و نمی‌تواند صفر باشد.
در این صورت حتما
$\langle v_1, v_1 \rangle = ||v_1||^2 > 0$
است.
حال استقرا می‌زنیم. فرض کنید می‌دانیم که برداری مثل
$w$
وجود دارد که برای
$\lbrace v_1, \cdots, v_{m-1} \rbrace$
داریم
$\forall i \in [1, m-1] ~ \langle w, v_i \rangle > 0$
است.
فرض کنید که می‌خواهیم که بردار مستقل خطی
$v_m$
از بقیه‌ی بردار‌های دیگر را به مجموعه اضافه کنیم.
همچنین به بردار
$w$
یک ضریبی از
$v_m$
را اضافه می‌کنیم. دقت کنید که این بردار از بقیه بردار‌ها مستقل خطی است پس عملا این قسمت اضافه شده را نمی‌توان
از بردار‌های قبلی بدست آورد.
حال باید ثابت کنیم که
$\forall i \in [1, m] ~ \langle w + av_m, v_i \rangle > 0$
برقرار است.
برای تمامی
$i \in [1, m-1]$
دقت کنید که اصلا
$v_m$
اثری بر روی آن‌ها ندارد چرا که مستقل خطی از بقیه‌ی آن‌ها است و طبق پایه‌ی استقرا
$\langle w + av_m, v_i \rangle > 0$
است. حال
$\langle w + av_m, v_m \rangle > 0$
را بررسی می‌کنیم.
دقت کنید که می‌توان ضرب داخلی را باز کرد به صورت زیر:
$\langle w + av_m, v_m \rangle = \langle w, v_m \rangle + \langle av_m, v_m \rangle$.
حال کافی است که اینقدر
$a$
را بزرگ انتخاب کنیم که
$\langle w, v_m \rangle + a\langle v_m, v_m \rangle > 0$
(دقت کنید که همان طور که در قبل نیز گفته شده بود
$\langle v_m, v_m \rangle > 0$
است.)

