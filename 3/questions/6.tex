\smalltitle{سوال 6}\\
فرض می‌کنیم که ماتریس تبدیل
$T$، $A$
است. آنگاه داریم:
\begin{gather*}
    T(u) \cdot T(w) = (Au)^T(Aw) = u^TA^TAw
\end{gather*}
حال ثابت می‌کنیم که
$A^TA = I$
است. برای این کار کافی است نشان دهیم که تمام عضو‌های روی قطر اصلی حاصل یک است و بقیه‌ی درایه‌ها صفر.
برای قطر اصلی دقت کنید که عملا جواب درایه بر روی قطر اصلی برابر است با ضرب داخلی هر ستون ماتریس
$A$
در خودش. می‌دانیم که ستون‌های این ماتریس یکه هستند پس
$A_i \cdot A_i = ||A_i||^2 = 1$
است. حال برای بقیه‌ی درایه‌ها دقت کنید که عملا داریم:
$A_i \cdot A_j (i \neq j)$
همچنین طبق فرض سوال این ستون‌ها متعامد هستند پس جواب ضرب داخلی مورد نظر برابر 0 است.
پس
$A^TA = I$
است. این نشان می‌دهد که
$u^TA^TAw = u^T I w = u \cdot w$
است.

\noindent
حال برای طرف دیگر قضیه این موضوع را اثبات می‌کنیم. فرض کنیم می‌دانیم که:
$T(u) \cdot T(w) = u \cdot w$
است. طبق تعریف داریم:
\begin{align*}
    u \cdot w & = T(u) \cdot T(w)\\
    & = (Au)^T(Aw)\\
    & = u^TA^TAw
\end{align*}
از این عبارت مشخص است که باید
$A^TA = I$
باشد. این نشان می‌دهد که باید ضرب داخلی هر ستون
$A$
در خودش برابر یک باشد که نشان می‌دهد اندازه‌ی بردار هر ستون برابر 1 است.
حال دقت کنید که هر دو ستون غیر یکسانی که در هم ضرب داخلی شده‌اند، برابر 0 شده است.
این نشان می‌دهد که هر دو ستون با هم دیگر متعامد هستند و فرض مسئله اثبات می‌شود.