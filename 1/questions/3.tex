\smalltitle{سوال 3}

\noindent
زمانی این تساوی برقرار نیست که یکی از عنصر‌های
$b$
برابر 0 باشد. در این صورت عنصر متناظر در
$a$ و $c$
می‌تواند متفاوت باشد ولی در ضرب
\lr{hadamard}
برابر صفر است.
حال اگر تنها یکی از درایه‌های بردار برابر صفر باشد برای نرم آن داریم:
\begin{equation*}
    ||b||^2 = \underbrace{b_1^2 + \ldots + b_{i-1}^2 + b_{i+1}^2 + \ldots + b_{n}^2}_{n-1}
    \le n-1
\end{equation*}
چرا که هر کدام از
$b_i^2$ها
حداکثر می‌توانند یک باشند.
پس فرض سوال نقض شد و ممکن نیست که
$b$
درایه‌ی صفر داشته باشد.
پس تساوی داده شده درست است چرا که:
\begin{equation*}
b_i \times a_i = b_i \times c_i \iff a_i = c_i (b_i \neq 0)
\end{equation*}