% !TEX program = xelatex
\documentclass[]{article}
\usepackage{graphicx}
\usepackage{float}
\usepackage[svgnames]{xcolor} 
\usepackage{fancyhdr}
\usepackage{tocloft}
\usepackage[hidelinks]{hyperref}
\usepackage[shortlabels]{enumitem}
\usepackage[many]{tcolorbox}
\usepackage{listings }
\usepackage[a4paper, total={6in, 8in}]{geometry}
\usepackage{afterpage}
\usepackage{amssymb}
\usepackage{pdflscape}
\usepackage{textcomp}
\usepackage{xecolor}
\usepackage{rotating}
\usepackage[Kashida=on,KashidaXBFix=on]{xepersian}
\usepackage[T1]{fontenc}
\usepackage{tikz}
\usepackage[utf8]{inputenc}
\usepackage{PTSerif} 
\usepackage{seqsplit}
\usepackage{changepage}
\usepackage{listings}
\usepackage{xcolor}
\usepackage{sectsty}
\settextfont[Scale=1]{XB Niloofar}
\begin{document}
{
	\thispagestyle{fancy}
	\fancyhf{}
	\fancyfoot{}
	\cfoot{\thepage}
	\lhead{99171333}
	\rhead{هیربد بهنام}
	\chead{تمرین کلاسی جبرخطی}
	\renewcommand{\headrulewidth}{2pt}
	\KashidaOff
}
\noindent
در ابتدا دقت کنید که می‌توان هر تابع از $R$ به $R$ را به یک بردار به شکل
$\begin{bmatrix}x \\ y\end{bmatrix}$
تصور کرد.
حال باید سه شرط زیر را برای
$W$
بررسی کنیم:
\begin{enumerate}
	\item بردار 0 در $W$ قرار دارد.
این موضوع به راحتی قابل نشان دادن است. کافی است که تابع
$f(x) = 0$
را در نظر بگیریم.
	\item به ازای هر $u$ و $v$ در $W$، $v + u$ نیز در $W$ قرار دارد.
دقت کنید که در صورتی که بخواهیم
$f_1$ و $f_2$
را از $W$ انتخاب کنیم، هر کدام از آن‌ها یک حد بالا و پایین دارند.
کافی است که
$M$ تابع $f_1 + f_2$
را برابر
$\operatorname{max}(|\operatorname{max}(f_1) + \operatorname{max}(f_2)|,|\operatorname{min}(f_1) + \operatorname{min}(f_2)|)$
قرار می‌دهیم و عنصر نشان دهنده‌ی $y$ آنرا با همدیگر جمع می‌کنیم.
پس با این توصیف
$f_1 + f_2 \in W$
است.
خلاصه: اگر دو تابعی که
\lr{bound}
داشته باشند را با هم جمع کنیم، تابعی به وجود می‌آید که
\lr{bound}
دارد.
	\item باید نشان دهیم به ازای هر $f$، $cf \in W$ است.
برای نشان دادن کافی است که
$M_f$ و هر نقطه‌ی محور $y$ را ضرب در $c$ کنیم.
\end{enumerate}

\end{document}