\smalltitle{سوال 4}\\
\begin{enumerate}[wide, labelwidth=!, labelindent=0pt]
\item
\begin{equation*}
\operatorname{avg}(v) = \frac{\sum v_i}{n} \implies
\operatorname{avg}(\alpha v + \beta 1) = \frac{\sum (\alpha v_i + \beta)}{n} = \frac{\alpha(\sum v_i) + n\beta}{n}
= \alpha \operatorname{avg}(v) + \beta
\end{equation*}
\item
\begin{gather*}
\operatorname{std}(v) = \sqrt{\frac{\sum(v_i - \operatorname{avg}(v))^2}{n}} \implies
\operatorname{avg}(\alpha v + \beta 1) = \sqrt{\frac{\sum((\alpha v_i + \beta) - (\alpha \operatorname{avg}(v) + \beta))^2}{n}} =\\
\sqrt{\frac{\alpha^2\sum(v_i - \operatorname{avg}(v))^2}{n}} = 
|\alpha|\sqrt{\frac{\sum((v_i - \operatorname{avg}(v)))^2}{n}} = |\alpha|\operatorname{std}(v)
\end{gather*}
\item از برهان خلف استفاده می‌کنیم. فرض می‌کنیم که تمامی عناصر از
$v_{\operatorname{rms}}$
کمترند. پس داریم:
\begin{gather*}
|a_i| < v_{\operatorname{rms}} \implies \sum a_i^2 < n ~ v_{\operatorname{rms}}^2 \implies
\sqrt{\frac{\sum a_i^2}{n}} < |v_{\operatorname{rms}}|
\end{gather*}
که این تناقض است! چرا که باید به جای نامساوی مساوی در می‌آمد. پس فرض خلف باطل است.
\end{enumerate}