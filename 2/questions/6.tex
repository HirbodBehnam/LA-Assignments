\\\smalltitle{سوال 6}\\
مربع فاصله‌ی دو بردار را می‌توان به صورت زیر نوشت:
$||y^*-x||^2=||y^*||^2+||x||^2-2 ||y^*||~||x|| \cos \theta$
که
$\theta$
زاویه‌ی بین دو بردار است.
حال دقت کنید که در تمامی
$y^*$ها،
$||x||$ ثابت است و
$||y^*||=1$ است که آن هم ثابت است.
حال دقت کنید که پس تنها
$\cos \theta$
متغیر است. حال دقت کنید که هر چه قدر زاویه بین دوبردار کوچک‌تر باشد
$\cos \theta$
عدد بیشتری است. پس عبارت بزرگتری از فاصله‌ی بین دو بردار کم می‌شود.
این یعنی هر چه قدر زاویه‌ی بین دو بردار کمتر باشه، فاصله‌ی بین آن دو نیز کمتر است.
(البته در این سوال!)
همچنین دقت کنید که با استدلال مشابه می‌توان به این نتیجه رسید که هر چه قدر
فاصله‌ی دو بردار کمتر باشد، زاویه‌ بین آن دو نیز کمتر است.