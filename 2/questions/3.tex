\\\smalltitle{سوال 3}
\\\noindent
در ابتدا باید
$\alpha$
را بدست آوریم. برای این کار از عملیات سطری مقدماتی استفاده می‌کنیم.
\begin{align*}
    \begin{array}{r@{\mskip\thickmuskip}l}
		0 &= 2x + 2y + 3z\\
		-4 &= 4x + 8y + 12z\\
		4 &= 6x + 2y + \alpha z
    \end{array}
    \implies
    \begin{array}{r@{\mskip\thickmuskip}l}
        0 &= 2x + 2y + 3z\\
        -4 &= 0x + 4y + 6z\\
        4 &= 0x - 4y + (\alpha - 9)z
    \end{array}
    \implies
    \begin{array}{r@{\mskip\thickmuskip}l}
        0 &= 2x + 2y + 3z\\
        -4 &= 0x + 4y + 6z\\
        0 &= 0x + 0y + (\alpha - 3)z
    \end{array}
\end{align*}
\begin{enumerate}[wide, labelwidth=!, labelindent=0pt]
    \item به ازای تمامی اعداد سازگار است چرا که همیشه می‌توان $z=0$ قرار داد.
    \item به ازای تمامی اعداد به غیر از $\alpha = 3$ جواب یکتا دارد.
در این حالت همیشه
$z = 0$
است. پس دستگاه زیر را حل می‌کنیم:
\begin{align*}
    \begin{array}{r@{\mskip\thickmuskip}l}
		0 &= 2x + 2y\\
		-4 &= 4x + 8y\\
    \end{array}
    \implies
    \begin{array}{r@{\mskip\thickmuskip}l}
        x = 1\\
        y = -1
    \end{array}
\end{align*}
    \item زمانی که $\alpha = 3$ باشد می‌توان به جای $z$ هر عددی را قرار داد. حال دستگاه زیر را بر حسب $x, y$ حل می‌کنیم.
\begin{align*}
    \begin{array}{r@{\mskip\thickmuskip}l}
		-3z &= 2x + 2y\\
        -4 - 6z &= 4y
    \end{array}
    \implies
    \begin{array}{r@{\mskip\thickmuskip}l}
        x &= 1\\
        y &= -\frac{2 + 3z}{2}
    \end{array}
\end{align*}
\end{enumerate}