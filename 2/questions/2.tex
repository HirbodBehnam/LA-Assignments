\smalltitle{سوال 2}\\
در صورتی که نقطه‌ی
$a$ درون $S$
قرار داشته باشد، کافی است که
$b = a$
انتخاب کنیم. در این حالت
$||a-b|| = 0$
می‌شود.
حال اگر
$a \not\subseteq S$
باشد، کافی است که نزدیک ترین نقطه بر روی
$S$ به $a$
را در نظر بگیریم.
\\\noindent
حال باید نشان دهیم که فقط یه نقطه با کمترین فاصله بر روی
$S$
وجود دارد. در صورتی که دو نقطه با کمترین فاصله داشته باشیم باید آن دو بر روی یک دایره قرار دارند
با مرکز
$a$.
حال اگر پاره خط واصل این دو نقطه را رسم کنیم، آن درون دایره می‌افتد و فاصله‌ی از تا
$a$
کمتر می‌شود.
ولی از طرفی می‌دانستیم که شکل
$S$، \lr{convex}
است. این نشان می‌دهد که این خط باید وجود می‌داشت چرا که در این صورت پاره خطی وجود دارد که دو سر آن دو
نقطه‌ی درون
$S$
است ولی آن پاره خط از خارج
$S$ رد می‌شود.
پس دو نقطه وجود ندارد و فقط یک نقطه وجود دارد که در شرط مسئله صدق می‌کند.