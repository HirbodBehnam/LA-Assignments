\smalltitle{سوال 5}\\
ماتریس دستگاه را تشکیل می‌دهیم و شروع به حل سوال می‌کنیم.
\begin{gather*}
    \begin{bmatrix}
        4 & -2 & -1 & 5 \\
        -1 & -1 & -1 & 2 \\
        1 & 1 & -1 & 0 \\
        1 & 3 & -1 & 3
    \end{bmatrix}
\end{gather*}
حال به صورت موقت از ردیف آخر صرف نظر می‌کنیم.
\begin{gather*}
    \begin{bmatrix}
        4 & -2 & -1 & 5 \\
        -1 & -1 & -1 & 2 \\
        1 & 1 & -1 & 0
    \end{bmatrix}
    \implies
    \begin{bmatrix}
        1 & -5 & -4 & 11 \\
        -1 & -1 & -1 & 2 \\
        0 & 0 & -2 & 2
    \end{bmatrix}
    \implies
    \begin{bmatrix}
        1 & -5 & -4 & 11 \\
        0 & -6 & -5 & 13 \\
        0 & 0 & 1 & -1
    \end{bmatrix}
    \implies
    \begin{bmatrix}
        1 & -5 & 0 & 7 \\
        0 & -6 & 0 & 8 \\
        0 & 0 & 1 & -1
    \end{bmatrix}\\
    \implies
    \begin{bmatrix}
        1 & 0 & 0 & \frac{1}{3} \\
        0 & 1 & 0 & \frac{-4}{3} \\
        0 & 0 & 1 & -1
    \end{bmatrix}
    \implies \quad
    \left\{\begin{array}{r@{\mskip\thickmuskip}l}
        a = \frac{1}{3} \\
        b = \frac{-4}{3} \\
        c = -1
    \end{array}\right.
\end{gather*}
پس در حال حاضر تابع ما به صورت
$f(x,n) = \frac{x^n}{3}-\frac{4x}{3}+1$
است. کافی است که حالا سطر آخر که حذف کردیم را تست کنیم که آیا در این تابع صدق می‌کند یا خیر:
\begin{gather*}
    f(3,0) = \frac{1}{3}-\frac{12}{3}+1=\frac{14}{3} \neq 3
\end{gather*}
پس همچین تابعی وجود ندارد.