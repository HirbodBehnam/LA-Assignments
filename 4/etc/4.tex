% !TEX program = xelatex
\documentclass[]{article}
\usepackage{graphicx}
\usepackage{float}
\usepackage[svgnames]{xcolor} 
\usepackage{fancyhdr}
\usepackage{tocloft}
\usepackage[hidelinks]{hyperref}
\usepackage[shortlabels]{enumitem}
\usepackage[many]{tcolorbox}
\usepackage{listings }
\usepackage[a4paper, total={6in, 8in}]{geometry}
\usepackage{afterpage}
\usepackage{amssymb}
\usepackage{pdflscape}
\usepackage{textcomp}
\usepackage{xecolor}
\usepackage{rotating}
\usepackage[Kashida=on,KashidaXBFix=on]{xepersian}
\usepackage[T1]{fontenc}
\usepackage{tikz}
\usepackage[utf8]{inputenc}
\usepackage{PTSerif} 
\usepackage{seqsplit}
\usepackage{changepage}
\usepackage{listings}
\usepackage{xcolor}
\usepackage{sectsty}
\settextfont[Scale=1]{XB Niloofar}
\begin{document}
{
	\thispagestyle{fancy}
	\fancyhf{}
	\fancyfoot{}
	\cfoot{\thepage}
	\lhead{99171333}
	\rhead{هیربد بهنام}
	\chead{تمرین کلاسی جبرخطی}
	\renewcommand{\headrulewidth}{2pt}
	\KashidaOff
}
\noindent % https://math.stackexchange.com/a/2133979/424863
فرض کنید که
$B = P^{-1}AP$
است. می‌خواهیم ثابت کنیم
$\operatorname{rank}(A) = \operatorname{rank}(B)$
است. برای شروع ثابت می‌کنیم که اگر
$A$
وارون پذیر باشد آنگاه
$\operatorname{rank}(AB) = \operatorname{rank}(B)$
است.

\noindent
فرض کنید که
$v \in \operatorname{nullspace}(B)$
باشد. پس طبق تعریف مشخص است که
$Bv = 0$
است. در صورتی که از چپ یک
$A$
را در معادله ضرب کنیم داریم که
$ABv = ABv = 0$
است. این نشان می‌‌دهد که
$v \in \operatorname{nullspace}(AB)$
نیز است. پس داریم:
$\operatorname{nullspace}(B) \subseteq \operatorname{nullspace}(AB)$
است. حال از طرفی دیگر فرض کنید که داریم
$u \in \operatorname{nullspace}(AB)$
است. آنگاه طبق تعریف
$ABu = 0$
است. می‌دانیم که از آنجا که
$A$
وارون پذیر است
$A^{-1}$
وجود دارد. در صورتی که این ماتریس را از چپ در معادله ضرب کنیم داریم که
$Bu = 0$
است. این نشان می‌دهد که
$u \in \operatorname{nullspace}(B)$
است. پس می‌توان نتیجه گرفت
$\operatorname{nullspace}(AB) \subseteq \operatorname{nullspace}(B)$
است. این دو معدله بدست آمده نشان می‌دهد که
$\operatorname{nullspace}(B) = \operatorname{nullspace}(AB)$
است پس از قضیه‌ی رنک می‌توان نتیجه گرفت
$\operatorname{rank}(AB) = \operatorname{rank}(B)$
است.

\noindent
حال ثابت می‌کنیم که اگر
$B$
وارون پذیر باشد آنگاه 
$\operatorname{rank}(AB) = \operatorname{rank}(A)$
است. برای نشان دادن این موضوع از قسمت قبل استفاده می‌کنیم:
\begin{gather*}
	\operatorname{rank}(AB) = \operatorname{rank}((AB)^T) = \operatorname{rank}(B^TA^T) = \operatorname{rank}(A^T) = \operatorname{rank}(A)
\end{gather*}
دقت کنید که اگر
$B$
وارون پذیر باشد آنگاه
$B^T$
نیز وارون پذیر است.

\noindent
حال به مسئله‌ی اصلی بر میگردیم. مشخص است که اگر
$B = P^{-1}AP \implies PB = AP$
است. حال در صورتی که از دو طرف رنک بگیریم و
$P$
وارون پذیر است داریم:
$\operatorname{rank}(A) = \operatorname{rank}(B)$

\end{document}