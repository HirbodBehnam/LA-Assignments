\smalltitle{سوال 6}\\
\smalltitle{الف}
کافی است که هر ماتریس را به صورت
\lr{rref}
در بیاوریم و ببینیم که آیا تمام عناصر قطر اصلی برابر یک هستند یا خیر.
\begin{gather*}
    A_1 = \begin{bmatrix}
        2 & 1 & 3\\
        -1 & 0 & 4\\
        6 & 2 & -4
    \end{bmatrix}
    \sim
    \begin{bmatrix}
        1 & 0 & 0\\
        0 & 1 & 0\\
        0 & 0 & 1
    \end{bmatrix}
    \\
    A_2 = \begin{bmatrix}
        1 & 1 & 2\\
        -1 & 1 & 2\\
        4 & -1 & 2
    \end{bmatrix}
    \sim
    \begin{bmatrix}
        1 & 0 & 0\\
        0 & 1 & 0\\
        0 & 0 & 1
    \end{bmatrix}
    \\
    A_3 = \begin{bmatrix}
        1 & 2 & 6 & 1\\
        0 & -2 & -1 & 0\\
        -1 & 3 & 3 & 6\\
        4 & -2 & 1 & 1
    \end{bmatrix}
    \sim
    \begin{bmatrix}
        1 & 0 & 0 & 0\\
        0 & 1 & 0 & 0\\
        0 & 0 & 1 & 0\\
        0 & 0 & 0 & 1
    \end{bmatrix}
\end{gather*}
در نتیجه تمام ماتریس‌ها
\lr{full rank matrix}
هستند.
\\\smalltitle{ب}
\begin{gather*}
    A_1:~
    v_1 = \begin{bmatrix}
        2\\
        -1\\
        6 
    \end{bmatrix}
    ~ v_2 = \begin{bmatrix}
        1 \\
        0\\
        2  
    \end{bmatrix}
    ~ v_3 = \begin{bmatrix}
        1\\
        0\\
        2  
    \end{bmatrix}
    \implies
    e_1 = \begin{bmatrix}
        \frac{2}{\sqrt{41}}\\
        \frac{-1}{\sqrt{41}}\\
        \frac{6}{\sqrt{41}} 
    \end{bmatrix}
    ~ e_2 = \begin{bmatrix}
        \frac{13}{3\sqrt{41}}\\
        \frac{14}{3\sqrt{41}}\\
        \frac{-2}{3\sqrt{41}} 
    \end{bmatrix}
    ~ e_3 = \begin{bmatrix}
        \frac{2}{3}\\
        \frac{-2}{3}\\
        \frac{-1}{3} 
    \end{bmatrix}
    \\
    A_2:~
    v_1 = \begin{bmatrix}
        1\\
        -1\\
        4 
    \end{bmatrix}
    ~ v_2 = \begin{bmatrix}
        1\\
        1\\
        -1  
    \end{bmatrix}
    ~ v_3 = \begin{bmatrix}
        2\\
        2\\
        2  
    \end{bmatrix}
    \implies
    e_1 = \begin{bmatrix}
        \frac{1}{3\sqrt{2}}\\
        \frac{-1}{3\sqrt{2}}\\
        \frac{4}{3\sqrt{2}} 
    \end{bmatrix}
    ~ e_2 = \begin{bmatrix}
        \frac{11}{3\sqrt{19}}\\
        \frac{7}{3\sqrt{19}}\\
        \frac{-1}{3\sqrt{19}} 
    \end{bmatrix}
    ~ e_3 = \begin{bmatrix}
        \frac{-3}{\sqrt{38}}\\
        \frac{5}{\sqrt{38}}\\
        \frac{4}{\sqrt{38}} 
    \end{bmatrix}
    \\
    A_3:~
    v_1 = \begin{bmatrix}
        1\\
        0\\
        -1\\
        4 
    \end{bmatrix}
    ~ v_2 = \begin{bmatrix}
        2\\
        -2\\
        3\\
        -2  
    \end{bmatrix}
    ~ v_3 = \begin{bmatrix}
        6\\
        -1\\
        3\\
        1 
    \end{bmatrix}
    ~ v_4 = \begin{bmatrix}
        1\\
        0\\
        6\\
        1 
    \end{bmatrix}
    \implies
    e_1 = \begin{bmatrix}
        \frac{1}{3\sqrt{2}}\\
        0\\
        \frac{-1}{3\sqrt{2}}\\
        \frac{4}{3\sqrt{2}}
    \end{bmatrix}
    ~ e_2 = \begin{bmatrix}
        \frac{5}{\sqrt{66}}\\
        \frac{-8}{\sqrt{66}}\\
        \frac{5}{\sqrt{66}}\\
        0
    \end{bmatrix}
    ~ e_3 = \begin{bmatrix}
        \frac{94\sqrt{2}}{3\sqrt{4301}}\\
        \frac{65}{\sqrt{8602}}\\
        \frac{-16\sqrt{2}}{3\sqrt{4301}}\\
        \frac{-5\sqrt{11}}{3\sqrt{782}}
    \end{bmatrix}
    ~ e_4 = \begin{bmatrix}
        \frac{-8\sqrt{2}}{\sqrt{1173}}\\
        \frac{25}{\sqrt{2346}}\\
        \frac{6\sqrt{6}}{\sqrt{391}}\\
        \frac{13}{\sqrt{2346}}
    \end{bmatrix}
\end{gather*}