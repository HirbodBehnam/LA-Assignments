\smalltitle{سوال 4}\\
\smalltitle{الف}
می‌دانیم که می‌توان ماتریس‌های
\lr{full column rank}
را به صورت
$\begin{bmatrix}
    I \\ 0
\end{bmatrix}$
در اورد به کمک
\lr{rref}.
در این صورت با معادله‌ی
$\begin{bmatrix}
    I \\ 0
\end{bmatrix} x = b'$
سر و کار داریم.
این معادله دو حالت دارد. یا اینکه
$m-n$
عنصر پایین
$b'$
برابر 0 هستند یا خیر. در صورتی که برابر صفر باشند این معادله یک جواب یکتا دارد که برابر با
$n$
عنصر بالای
$b'$
است.
در غیر این صورت معادله جواب ندارد.
\\\smalltitle{ب}
با توجه به اینکه ماتریس وارون راست دارد، یکی از جواب‌های معادله برابر است با
$A^{-1}b$.
حال دقت کنید که از آن‌جا که
$m \neq n$
است، قطعا
$\operatorname{dim}(\operatorname{nullspace}(A)) > 0$
است. پس در نتیجه هر
$A^{-1}b + v$
که
$v \in \operatorname{nullspace}(A)$
یک جواب مسئله است. پس بی‌نهایت جواب دارد.


