\smalltitle{سوال 3}\\
در ابتدا ثابت می‌کنیم اگر
$A \in M_{mn} (\mathbb{R}), \operatorname{rank}(A) = r$
آنگاه می‌توان
$A$
را به صورت
$A = BC ~ (B \in M_{mr} (\mathbb{R}), C \in M_{rn} (\mathbb{R}))$
نوشت. از این اطلاعات می‌توان نتیجه گرفت که بُعد فضای ستونی ماتریس
$A$
برابر
$r$
است. پس تعداد پایه‌های این فضا برابر
$r$
است و آن‌ها را
$b_1, \cdots, b_r$
می‌نامیم. حال ماتریس
$B$
را به صورت
$B = [b_1, \cdots, b_r]$
در نظر می‌گیریم.
حال ماتریس
$r$
در
$n$
$C$
را طوری در نظر می‌گیریم که برای هر ستون
$i$ در $A$
داشته باشیم:
\begin{gather*}
    a_i = \sum_{j=1}^{r} c_{ji} b_j
\end{gather*}
در ادامه عبارت زیر را اثبات می‌کنیم:
\begin{gather*}
    \operatorname{rank}(AB) + n \geq \operatorname{rank}(A) + \operatorname{rank}(B)
\end{gather*}
برای این اثبات در ابتدا ثابت می‌کنیم که اگر
$A \in M_{mn} (\mathbb{R}), B \in M_{np} (\mathbb{R})$
باشد آنگاه داریم:
\begin{gather*}
    \operatorname{rank}(A) + \operatorname{rank}(B) \leq \operatorname{rank}(
    \begin{bmatrix}
        I_n & B \\
        A & 0
    \end{bmatrix})
\end{gather*}
برای نشان دادن این موضوع نشان می‌دهیم که امکان ندارد که رنک ماتریس نشان داده شده کمتر از رنک دو ماتریس دیگر شود.
برای این کار فرض کنید که
$\operatorname{rank}(A) = a, \operatorname{rank}(B) = b$
است و ستون‌های مستقل خطی هر کدام از ماتریس‌ها را
$U_1, \cdots, U_a$ و $V_1, \cdots, V_b$
می‌نامیم. حال شرط وابسته خطی بودن را می‌نویسیم:
(دقت کنید که
$e_i$
نشان دهنده‌ی ستون متناظر به ستون
$i$ام
مستقل خطی ماتریس
$A$
در
$I_n$
است.)
\begin{gather*}
    x_1
    \begin{bmatrix}
        e_1 \\ U_1
    \end{bmatrix}
    + \cdots +
    x_a
    \begin{bmatrix}
        e_a \\ U_a
    \end{bmatrix}
    +
    y_1
    \begin{bmatrix}
        V_1 \\ 0
    \end{bmatrix}
    + \cdots +
    y_b
    \begin{bmatrix}
        V_b \\ 0
    \end{bmatrix}
    = 0
\end{gather*}
حال دقت کنید که زمانی این معادله برقرار است که تمام
$x_i$ها
برابر صفر باشد و در نتیجه تمام
$y_i$ها
نیز برابر صفر باشند.
(دقت کنید که ستون‌ها مستقل خطی بودند.)
این نشان می‌دهد که ماتریس حاصل نمی‌تواند باعث شود که تعداد ستون‌های بیشتری از
$a + b$
در آن مستقل خطی شوند و حکم مسئله اثبات می‌شود.

حال می‌دانیم که عبارت زیر برقرار است:
\begin{gather*}
    \begin{bmatrix}
        I_n & 0\\
        -A & I_m
    \end{bmatrix}
    \begin{bmatrix}
        I_n & B\\
        A & 0
    \end{bmatrix}
    \begin{bmatrix}
        I_n & -B\\
        0 & I_p
    \end{bmatrix}
    =
    \begin{bmatrix}
        I_n & 0\\
        0 & -AB
    \end{bmatrix}
\end{gather*}
حال دقت کنید که جفت
$\begin{bmatrix}
    I_n & 0\\
    -A & I_m
\end{bmatrix}$
و
$\begin{bmatrix}
    I_n & -B\\
    0 & I_p
\end{bmatrix}$
وارون پذیر هستند چرا که مثلثی و بدون عضو 0 بر روی قطر اصلی هستند.
این موضوع نتیجه می‌دهد که
\begin{gather*}
    \operatorname{rank}(\begin{bmatrix}
        I_n & 0\\
        0 & -AB
    \end{bmatrix})
    =
    \operatorname{rank}(\begin{bmatrix}
        I_n & B\\
        A & 0
    \end{bmatrix})
\end{gather*}
حال این موضوع را در عبارت بدست آمده در قسمت قبل استفاده می‌کنیم:
\begin{gather*}
    \operatorname{rank}(A) + \operatorname{rank}(B) \leq \operatorname{rank}(
    \begin{bmatrix}
        I_n & B \\
        A & 0
    \end{bmatrix}) = 
    \operatorname{rank}(\begin{bmatrix}
        I_n & 0\\
        0 & -AB
    \end{bmatrix})
    = n + \operatorname{rank}(AB)
\end{gather*}

حال در مسئله‌ی اصلی فرض می‌کنیم که
$\operatorname{rank}(B) = r$
است. همچنین
$B$
را به صورت
$B = U_{nr} \times V_{rn}$
با توجه به اولین چیزی که اثبات شد تجزیه می‌کنیم.
حال از معادله‌ی قبل داریم:
\begin{gather*}
    \operatorname{rank}(ABC) + r = \operatorname{rank}((AU)(VC)) + r \geq \operatorname{rank}(AU) + \operatorname{rank}(VC)
\end{gather*}
حال می‌دانیم که
$\operatorname{rank}(AB) \leq \operatorname{rank}(A)$
و
$\operatorname{rank}(AB) \leq \operatorname{rank}(B)$
است پس داریم:
\begin{gather*}
    \operatorname{rank}(AU) + \operatorname{rank}(VC) \geq \operatorname{rank}(AUV) + \operatorname{rank}(UVC) 
    = \operatorname{rank}(AB) + \operatorname{rank}(BC)
\end{gather*}
پس در نتیجه داریم:
\begin{gather*}
    \operatorname{rank}(ABC) + r = \operatorname{rank}(ABC) + \operatorname{rank}(B) \geq \operatorname{rank}(AB) + \operatorname{rank}(BC)
\end{gather*}