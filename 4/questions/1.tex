\smalltitle{سوال 1}\\
\smalltitle{الف}\\
در ابتدا دقت کنید که در صورتی که
$i=j=k$
باشند داریم:
$xa_{ii} + ya_{ii} + za_{ii} = 0$
در نتیجه تمامی
$a_{ii}$ها
صفر هستند. پس تمام قطر اصلی صفر است.

\noindent
در مرحله‌ی بعدی تنها دو متغیر را مساوی هم قرار می‌دهیم و معادله‌های حاصل را می‌نویسیم.
\begin{gather*}
    \begin{array}{c}
        xa_{ii} + ya_{ij} + za_{ji} = 0\\
        xa_{ji} + ya_{ii} + za_{ij} = 0\\
        xa_{ij} + ya_{ji} + za_{ii} = 0
    \end{array}
    \stackrel{a_{ii} = 0}{\implies}
    \begin{array}{c}
        ya_{ij} + za_{ji} = 0\\
        xa_{ji} + za_{ij} = 0\\
        xa_{ij} + ya_{ji} = 0
    \end{array}
    \stackrel{\{\times y, \times z\}}{\implies} 
    \begin{array}{c}
        y^2a_{ij} + yza_{ji} = 0\\
        xza_{ij} + yza_{ji} = 0\\
    \end{array}
    \stackrel{(-)}{\implies}
    a_{ij} (y^2 - xz) = 0
\end{gather*}
دقت کنید که در اینجا دو حالت رخ می‌دهد. یکی اینکه تمامی درایه‌های ماتریس صفر باشند که در این صورت رنک
ماتریس برابر 0 می‌شود و حکم مسئله نقض نمی‌شود.
در غیر این صورت نتیجه می‌شود
$y^2 = xz$
است. دقت کنید که دقیقا در صورتی که بقیه‌ی
$x, y, z$
را در معادله‌های دیگر ضرب کنیم عبارت‌های مشابه نیز بدست می‌آوریم.
این عبارات عبارتند از:
\begin{gather*}
    x^2 = yz\\
    y^2 = xz\\
    z^2 = xy
\end{gather*}
دقت کنید که از این سه معادله می‌توان نتیجه گرفت که
$x = y = z \neq 0$
است. حال در معادله‌ی اصلی داریم:
$ya_{ij} + za_{ji} = 0 \implies x(a_{ij} + a_{ji}) = 0 \stackrel{x \neq 0}{\implies} a_{ij} = -a_{ji}$
این موضوع نشان می‌دهد که ماتریس ما پاد متقارن است.

\noindent
در مرحله‌ی آخر فرض کنید که
$i \neq j \neq k$
باشد. در این صورت داریم:
$x (a_{ij} + a_{jk} + a_{ki}) = 0 \stackrel{x \neq 0}{\implies} a_{ij} + a_{jk} + a_{ki} = 0$
پس در نتیجه ماتریس مورد نظر ما به صورت زیر است:
\begin{gather*}
    \begin{bmatrix}
        0 & b_1 & b_1 + b_2 & \cdots & \sum_{i=1}^{n-1} b_i \\
        -b_1 & 0 & b_2 & \cdots & \sum_{i=2}^{n-1} b_i\\
        -b_1 & -b_1 - b_2 & 0 & \cdots & \sum_{i=3}^{n-1} b_i\\
        \vdots & \vdots & \vdots & \ddots & \vdots\\
        -\sum_{i=1}^{n-1} b_i & -\sum_{i=2}^{n-1} b_i & \cdots & \cdots & 0
    \end{bmatrix}
\end{gather*}
\\\smalltitle{ب}\\
با توجه به مطالب فوق می‌توان ماتریس زیر را مثال زد:
\begin{gather*}
    b_1 = 1, b_2 = 2\\
    \begin{bmatrix}
        0 & 1 & 3\\
        -1 & 0 & 2\\
        -3 & -2 & 0
    \end{bmatrix}
\end{gather*}