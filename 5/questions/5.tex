\smalltitle{سوال 5}\\
\smalltitle{الف}
برای چک کردن اینکه آیا برداری، بردار ویژه‌ی ماتریس است یا خیر باید چک کنیم که آیا
$\lambda$ای
وجود دارد که
$Au = \lambda u$
باشد یا خیر.
\begin{gather*}
    u = Av + v \implies A(Av + v) = A^2v + Av = v + Av\\
    u = Av - v \implies A(Av - v) = A^2v - Av = v - Av
\end{gather*}
همان طور که مشاهده می‌کند در صورتی که
$\lambda$
را به ترتیب یک و منفی یک در نظر بگیریم تساوی برقرار می‌شود. پس دو بردار داده شده، بردار ویژه‌ی ماتریس
$A$
هستند.
\\\smalltitle{ب}
در می‌دانیم از قسمت قبل که برای هر بردار
$v$، آنگاه
$Av - v$ و $Av + v$
دو بردار ویژه ماتریس
$A$
هستند. حال دقت کنید که یکی از ترکیب خطی‌های بردار‌ها برابر است با
$\frac{Av+v}{2}-\frac{Av-v}{2}=v$
است. این موضوع به راحتی نشان می‌دهد که در صورتی که
$u=v$ انتخاب کنیم،
به راحتی می‌توان به کمک این ترکیب خطی
$u$ را ساخت.
\\\smalltitle{ج}
فرض کنید که ماتریس
$A$
قابلیت قطری شدن داشته باشد. بدین منظور که
$A = PDP^{-1}$
باشد. پس در نتیجه
$P^{-1}AP = D$
است. حال دقت کنید که:
\begin{gather*}
    P^{-1}(A+I)P = P^{-1}AP + P^{-1}IP = P^{-1}AP + I
\end{gather*}
است. مشخص است که از آنجا که هم
$I$
قطری است و هم
$P^{-1}AP$
جمع آنها نیز قطری است. پس ماتریس
$A+I$
قطری پذیر است.
\\\smalltitle{د}
دقت کنید که
$A^2 = I \implies A = A^{-1}$
است. پس باید چک کنیم که آیا
$2A + I$
قطری پذیر است یا خیر.
از قسمت ب همین سوال می‌دانیم که هر برداری را با بردار ویژه‌های ماتریس
$A$
ساخت. این بدین معنی است که بردار ویژه‌های
$A$
مستقل خطی هستند و در نتیجه
$A$
قطری پذیر است. مشخص است که
$2A$
نیز قطری پذیر است. در نهایت طبق قسمت ج می‌دانیم که جمع یک ماتریس قطری پذیر با
$I$
نیز قطری پذیر است. پس
$2A + I = A + A^{-1} + I$
قطری پذیر است.




