\smalltitle{سوال 1}\\
% https://math.stackexchange.com/questions/1184825/upper-triangular-block-matrix-determinant-by-induction
از استقرا استفاده می‌کنیم. در پایه‌ی فرض مشخص است که
$\det(A) = \det(A_1)$
است. حال برای حالت کلی تر ماتریس زیر را در نظر بگیرید:
\begin{gather*}
    X =
    \left[
        \begin{array}{ccccc|c}
            A_1 & * & \cdots & \cdots & * & *\\
            0 & A_2 & * & \cdots & * & *\\
            0 & 0 & A_3 & \ddots & \vdots & *\\
            \vdots & \ddots & \ddots & \ddots & \vdots & *\\
            0 & \cdots & \cdots & \cdots & A_{n-1} & *\\
            \hline
            0 & \cdots & \cdots & \cdots & 0 & A_n
        \end{array}
    \right]
    = \begin{bmatrix}
        A & C\\
        0 & B
    \end{bmatrix}
\end{gather*}
دقت کنید که ماتریس سمت راست یک ماتریس بلوکی است و همچنین
$B$
نیز یک ماتریس بلوکی و مربعی است. طبق استقرا می‌دانیم که
$\det(A) = \prod_{i=1}^{n-1} \det(A_i)$
است. حال باید ثابت کنیم که
$\det(X) = \det(A) \det(B)$
است. در صورتی که
$A$
وارون پذیر نباشد، مشخص است که یک سری ستون وابسته خطی دارد. این موضوع نشان می‌دهد که حداقل دو تا از ستون‌های
سمت چپ
$X$
وابسته خطی هستند. این موضوع نشان می‌دهد که ماتریس
$X$
هم وارون پذیر نیست اگر
$A$
وارون پذیر نباشد.
پس طبق خواص دترمینان می‌دانیم که
$\det(A) = 0 \implies \det(X) = 0$.
در این حالت فرض عوض نمی‌شود و بر قرار می‌ماند.
در صورتی که
$\det(A) \neq 0$
باشد داریم:
\begin{gather*}
    \begin{bmatrix} A & C \\ 0 & B\end{bmatrix} = \begin{bmatrix} A & 0 \\ 0 & I_n\end{bmatrix} \, \begin{bmatrix} I_m & A^{-1}C \\ 0 & B\end{bmatrix} \implies\\
    \det(\begin{bmatrix} A & C \\ 0 & B\end{bmatrix}) = \det(\begin{bmatrix} A & 0 \\ 0 & I_n\end{bmatrix}) \times \det(\begin{bmatrix} I_m & A^{-1}C \\ 0 & B\end{bmatrix})
\end{gather*}
برای محاسبه‌ی
$\det(\begin{bmatrix} A & 0 \\ 0 & I_n\end{bmatrix})$
کافی است که دترمینان را از ستون سمت راست به چپ باز کنیم. در این صورت عملا داریم:
\begin{gather*}
    \det(\begin{bmatrix} A & 0 \\ 0 & I_n\end{bmatrix}) = \underbrace{1 \times \cdots \times 1}_{n ~ \text{times}} \times \det(A)
\end{gather*}
به طریق مشابه اگر بخواهیم
$\det(\begin{bmatrix} I_m & A^{-1}C \\ 0 & B\end{bmatrix})$
را حساب کنیم در صورتی که دترمینان را بر حسب ستون‌ها از چپ به راست باز کنیم، جواب برابر
$\det(B)$
می‌شود. پس در نتیجه:
\begin{gather*}
    \det(X) = \det(A)\det(B) = \det(A_1) \det(A_2) \cdots \det(A_n)
\end{gather*}