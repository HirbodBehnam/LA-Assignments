\smalltitle{سوال 1}\\
\smalltitle{الف}
در روش کمترین مربعات ما به دنبال کمینه کردن
$\left\|Ax - b\right\|$
هستیم. می‌دانیم که
$Ax$
(به صورت کلی)
فضای ستونی
$A$
را
\lr{span}
می‌کند پس عبارت
"سطری"
باید به 
"ستونی"
تغییر پیدا کند و عبارت غلط است.
\\\smalltitle{ب}
درست است. همان طور که گفته شد ما به دنبال کمینه کردن
$\left\|Ax - b\right\|$
هستیم که هم معنی است با نقطه‌ای از فضای ستونی
$A$
که کمترین فاصله با
$b$
را دارد. همچنین می‌دانیم که نزدیک ترین نقطه به برداری در یک زیرفضا از عمود کردن آن بردار بر زیرفضا بدست می‌آید.
در این این نقطه دقیقا همان
$\hat{b}$
است. پس عبارت درست است.
دقت کنید که در صورتی که
$b \in Ax$
باشد آنگاه
$b = \hat{b}$
می‌شود که باز هم پاسخ درست را به ما می‌دهد.
\\\smalltitle{ج}
دقت کنید که اگر
$b \in Ax$
باشد آنگاه یعنی
$\left\|Ax - b\right\| = 0$
است. پس عبارت درست است.
\\\smalltitle{د}
می‌دانیم که معادله خط به صورت
$y = mx + c$
است. در این سوال باید فرض کنیم که
$y = mx$
است چرا که گفته شده است که خط مبدا گذر است. حال برای کمترین مربعات باید
$\left\|Xm - Y\right\|$
را مینیمم کنیم که
$X$ و $Y$
بردار‌های
$x$ و $y$
نقاط داده شده هستند. این برابر است با مینیمم کردن
$\sum_{i=1}^n (mx_i - y_i)^2$
بدست می‌آید. حال داریم:
\begin{gather*}
    \frac{\partial}{\partial m} \sum_{i=1}^n (mx_i - y_i)^2 = \sum_{i=1}^n 2 x_i (mx_i - y_i) = 0\\
    m = \frac{\sum_{i=1}^n x_i y_i}{\sum_{i=1}^n x_i^2}
\end{gather*}
پس درست است.



